% THIS IS SIGPROC-SP.TEX - VERSION 3.1
% WORKS WITH V3.2SP OF ACM_PROC_ARTICLE-SP.CLS
% APRIL 2009
%
% Questions regarding SIGS should be sent to
% Adrienne Griscti ---> griscti@acm.org
%
% Questions/suggestions regarding the guidelines, .tex and .cls files, etc. to
% Gerald Murray ---> murray@hq.acm.org
%
% For tracking purposes - this is V3.1SP - APRIL 2009
%
% Copied from https://github.com/heathermiller/papers-documents/tree/master/rem2013

\documentclass{support/acm_proc_article-sp}
\usepackage{listings}
\usepackage{url}
\usepackage{support/bcprules}
\usepackage{support/prooftree}
\usepackage{support/math}
\usepackage{multicol}
\usepackage{caption}
\usepackage{subcaption}
\usepackage[normalem]{ulem}
\usepackage{color}
\usepackage{graphicx}
\usepackage{hyperref}

\renewcommand{\thesubsection}{\thesection.\alph{subsection}}

\definecolor{blue}{rgb}{0,0,0.5}
\definecolor{red}{rgb}{0.6,0,0}
\definecolor{green}{rgb}{0,0.5,0}
\definecolor{grey}{rgb}{0.2,0.2,0.2}

\lstdefinelanguage{Python}{
keywords={typeof, null, catch, switch, in, int, str, float, self, from, import},
keywordstyle=\color{blue}\bfseries,
ndkeywords={boolean, throw, import},
ndkeywords={return, class, if ,elif, endif, while, do, else, True, False , catch, def},
ndkeywordstyle=\color{blue}\bfseries,
identifierstyle=\color{grey},
sensitive=false,
comment=[l]{\#},
morecomment=[s]{/*}{*/},
commentstyle=\color{green}\ttfamily,
stringstyle=\color{green}\ttfamily,
}

\lstset{language=Python}

\clubpenalty = 10000
\widowpenalty = 10000
\displaywidowpenalty = 10000

% Remove copyright space
\makeatletter
\def\@copyrightspace{\relax}
\makeatother

% comments and notes
\newcommand{\comment}[1]{}
\newcommand{\note}[1]{{\bf $\clubsuit$ #1 $\spadesuit$}}
\newcommand{\ifreport}[1]{#1}
%\newcommand{\ifreport}[1]{}

\newcommand{\todo}{{\bf \colorbox{red}{\color{white}TODO:}}}
\newcommand{\ie}{{\em i.e.,~}}
\newcommand{\eg}{{\em e.g.,~}}
\newcommand{\term}[1]{\mbox{\texttt{#1}}}
\newcommand{\itl}[1]{\mbox{\textit{#1}}}

% commas and semicolons
\newcommand{\comma}{,\,}
\newcommand{\commadots}{\comma \ldots \comma}
\newcommand{\semi}{;\mbox{;};}
\newcommand{\semidots}{\semi \ldots \semi}

% spacing
\newcommand{\gap}{\quad\quad}
\newcommand{\biggap}{\quad\quad\quad}
\newcommand{\nextline}{\\ \\}
\newcommand{\htabwidth}{0.5cm}
\newcommand{\tabwidth}{1cm}
\newcommand{\htab}{\hspace{\htabwidth}}
\newcommand{\tab}{\hspace{\tabwidth}}
\newcommand{\linesep}{\ \hrulefill \ \smallskip}

\newcommand{\sectionline}{%
\nointerlineskip \vspace{\baselineskip}%
\hspace{\fill}\rule{0.5\linewidth}{.7pt}\hspace{\fill}%
\par\nointerlineskip \vspace{\baselineskip}
}

% figures
\newcommand{\figurebox}[1]
{\fbox{\begin{minipage}{\textwidth}
           #1 \medskip
\end{minipage}}}
\newcommand{\twofig}[3]
{\begin{figure*}[t]
     #3\ \hrulefill\
        \caption{\label{#1}#2}
\end{figure*}}
\newcommand{\boxfig}[3]
{\begin{figure*}
     \figurebox{#3\caption{\label{#1}#2}}
\end{figure*}}
\newcommand{\figref}[1]
{Figure~\ref{#1}}

% arrays
\newcommand{\ba}{\begin{array}}
\newcommand{\ea}{\end{array}}
\newcommand{\bda}{\[\ba}
\newcommand{\eda}{\ea\]}
\newcommand{\ei}{\end{array}}
\newcommand{\bcases}{\left\{\begin{array}{ll}}
\newcommand{\ecases}{\end{array}\right.}

\pagenumbering{arabic}
\begin{document}

    \title{Data Mining and Matrices (FSS18) \\ Assignment 1: Singular Value Decomposition}

    \numberofauthors{1}
    \author{
    \alignauthor
    Steffen Schmitz\\
    \affaddr{University of Mannheim}\\
    \affaddr{stefschm@mail.uni-mannheim.de}
    }

    \maketitle

    %%%%%%%%%%%%%%%%%%%%%%%%%%%%%%%%%%%%%%%%%%%%%%%%%%%%
    %%
    %% 1) DATASET STATISTICS
    %%
    %%%%%%%%%%%%%%%%%%%%%%%%%%%%%%%%%%%%%%%%%%%%%%%%%%%%

    \section{Dataset Statistics}
    \vspace{1.5\baselineskip}

    Explore and preprocess the dataset.

    %%%%%%%%%%%%%%%%%%%%%%%%%%%%%%%%%%%%%%%%%%%%%%%%%%%%
    %%
    %% 1.a) KERNEL DENSITY PLOT
    %%
    %%%%%%%%%%%%%%%%%%%%%%%%%%%%%%%%%%%%%%%%%%%%%%%%%%%%

    \subsection{Kernel Density Plot}
    \label{sec:kdp}
    \vspace{\baselineskip}

    \textbf{Task.} Look at the kernel density plot (code provided) of all features and discuss what you see (or don't see).

    We estimate the kernel density estimation with the gaussian\_kde method from the scipy.stats
    package\footnote{\href{https://docs.scipy.org/doc/scipy/reference/generated/scipy.stats.gaussian_kde.html}{https://docs.scipy.org/gaussian\_kde}}.
    It returns the estimated probability density function of each feature.
    As described in the documentation the plot may be "oversmoothed" what leads to non-zero results for $x < 0$,
    although the statistics (Output of Figure \ref{fig:describe}) show that all features have minimal values that are
    greater or equal to zero.

    %%%%%%%%%%%%%%%%%%%%%%%%%%%%%%%%%%%%%%%%%%%%%%%%%%%%
    %%
    %% BIBLIOGRAPHY
    %%
    %%%%%%%%%%%%%%%%%%%%%%%%%%%%%%%%%%%%%%%%%%%%%%%%%%%%

    \bibliographystyle{abbrv}
    \bibliography{support/bib}

\end{document}
