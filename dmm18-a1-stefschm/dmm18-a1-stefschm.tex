% THIS IS SIGPROC-SP.TEX - VERSION 3.1
% WORKS WITH V3.2SP OF ACM_PROC_ARTICLE-SP.CLS
% APRIL 2009
%
% Questions regarding SIGS should be sent to
% Adrienne Griscti ---> griscti@acm.org
%
% Questions/suggestions regarding the guidelines, .tex and .cls files, etc. to
% Gerald Murray ---> murray@hq.acm.org
%
% For tracking purposes - this is V3.1SP - APRIL 2009
%
% Copied from https://github.com/heathermiller/papers-documents/tree/master/rem2013

\documentclass{support/acm_proc_article-sp}
\input{support/praeambel}
\pagenumbering{arabic}
\begin{document}

    \title{Data Mining and Matrices (FSS18) \\ Assignment 1: Singular Value Decomposition}

    \numberofauthors{1}
    \author{
    \alignauthor
    Steffen Schmitz\\
    \affaddr{University of Mannheim}\\
    \affaddr{stefschm@mail.uni-mannheim.de}
    }

    \maketitle

    %%%%%%%%%%%%%%%%%%%%%%%%%%%%%%%%%%%%%%%%%%%%%%%%%%%%
    %%
    %% 1) DATASET STATISTICS
    %%
    %%%%%%%%%%%%%%%%%%%%%%%%%%%%%%%%%%%%%%%%%%%%%%%%%%%%

    \section{Dataset Statistics}
    \vspace{1.5\baselineskip}

    Explore and preprocess the dataset.

    %%%%%%%%%%%%%%%%%%%%%%%%%%%%%%%%%%%%%%%%%%%%%%%%%%%%
    %%
    %% 1.a) KERNEL DENSITY PLOT
    %%
    %%%%%%%%%%%%%%%%%%%%%%%%%%%%%%%%%%%%%%%%%%%%%%%%%%%%

    \subsection{Kernel Density Plot}
    \label{sec:kdp}
    \vspace{\baselineskip}

    \textbf{Task.} Look at the kernel density plot (code provided) of all features and discuss what you see (or don't see).

    We estimate the kernel density estimation with the gaussian\_kde method from the scipy.stats
    package\footnote{\href{https://docs.scipy.org/doc/scipy/reference/generated/scipy.stats.gaussian_kde.html}{https://docs.scipy.org/gaussian\_kde}}.
    It returns the estimated probability density function of each feature.
    As described in the documentation the plot may be "oversmoothed" what leads to non-zero results for $x < 0$,
    although the statistics (Output of Figure \ref{fig:describe}) show that all features have minimal values that are
    greater or equal to zero.

    %%%%%%%%%%%%%%%%%%%%%%%%%%%%%%%%%%%%%%%%%%%%%%%%%%%%
    %%
    %% BIBLIOGRAPHY
    %%
    %%%%%%%%%%%%%%%%%%%%%%%%%%%%%%%%%%%%%%%%%%%%%%%%%%%%

    \bibliographystyle{abbrv}
    \bibliography{support/bib}

\end{document}
